\documentclass[11pt, answers]{exam}
\usepackage[margin=1in]{geometry}
\usepackage{amsfonts, amsmath, amssymb, amsthm}
\usepackage{mathtools}
\usepackage{enumerate}
\usepackage{listings}
\usepackage{cancel}
\usepackage{hyperref}
\usepackage{booktabs}
\usepackage[boxed]{algorithm}
\usepackage[noend]{algpseudocode}
\usepackage{tikz}
\usepackage{float}

%
% Basic Document Settings
%

% \topmargin=-0.5in
% \evensidemargin=0in
% \oddsidemargin=0in
% \textwidth=6.5in
% \textheight=9.0in
% \headsep=0.25in

% \linespread{1.1}

\geometry{left=2cm,right=2cm,top=2cm,bottom=2cm}

\pagestyle{headandfoot}
\lhead{\hmwkClass\ : \hmwkType\ \#\hmwkNumber\ (Due \hmwkDue)}
\cfoot{\thepage}
% \renewcommand\headrulewidth{0.4pt}
% \renewcommand\footrulewidth{0.4pt}

\setlength\parindent{0pt}

%
% Create Problem Sections
%
\qformat{\hfill}

\newcommand{\hmwkType}{Electronic}
\newcommand{\hmwkNumber}{10}
\newcommand{\hmwkClass}{VE 492}
\newcommand{\hmwkDue}{July 31st, 2020 at 11:59pm}


%
% Title Page
%

\title{Homework \hmwkNumber\ \hmwkType}
\date{\hmwkDue}

%
% Various Helper Commands
%

% space of real numbers \R
\newcommand{\R}{\mathbb{R}}

% expected value \EX
\DeclareMathOperator{\EX}{\mathbb{E}}

% For partial derivatives \pderiv{}{}
\newcommand{\pderiv}[2]{\frac{\partial}{\partial #1} (#2)}

% argmax \argmax
\DeclareMathOperator*{\argmax}{arg\,max}

% sign \sign
\DeclareMathOperator{\sign}{sign}

% norm \norm{}
\DeclarePairedDelimiter{\norm}{\lVert}{\rVert}

% Keys
\newcommand{\key}[1]{\fbox{{\sc #1}}}
\newcommand{\ctrl}{\key{ctrl}--}
\newcommand{\shift}{\key{shift}--}
\newcommand{\run}{\key{run} \ }
\newcommand{\runkey}[1]{\run \key{#1}}
\newcommand{\extend}{\key{extend} \ }
\newcommand{\kkey}[1]{\key{k$_{#1}$}}

\begin{document}
\maketitle

%
% Question 1
%
\begin{questions}
\section{Propositional Logic 1}
\question

We ask a logician (who only tells the truth) about his sentimental life, and he answers the following two statements:
\begin{itemize}
	\item I love Ann or I love Beth.
	\item If I love Ann, then I love Beth.
\end{itemize}
What can we conclude? Answer the following questions by "yes", "no", "unsure".

\begin{enumerate}
\item Does he love Ann?
\item Does he love Beth?
\item Does he love both?
\end{enumerate}

\paragraph{Sample Answer:}
\begin{verbatim}
no,no,no
\end{verbatim}
\end{questions}


%
% Question 2
%
\begin{questions}
\section{Propositional Logic 2}
\question

Which of the following are correct?
\begin{enumerate}[a.]
\item $False \models True$.
\item $True \models False$.
\item $(A \wedge B) \models (A \Leftrightarrow B)$.
\item $A \Leftrightarrow B \models A \vee B$.
\item $A \Leftrightarrow B \models \neg A \vee B$.
\item $(A\wedge B) \Rightarrow C \models (A\Rightarrow C)\vee (B\Rightarrow C)$.
\item $(C\vee(\neg A\wedge\neg B))\equiv((A\Rightarrow C)\wedge(B\Rightarrow C)$.
\item $(A\vee B)\wedge(\neg C\vee\neg D\vee E)\models(A\vee B)$.
\item $(A\vee B)\wedge(\neg C\vee \neg D\vee E)\models (A\vee B)\wedge(\neg D\vee E)$.
\item $(A\vee B)\wedge \neg(A\Rightarrow B)$ is satisfiable.
\item $(A\Leftrightarrow B)\wedge(\neg A\vee B)$ is satisfiable.
\item $(A \Leftrightarrow B)\Leftrightarrow C$ has the same number of models as $A\Leftrightarrow B$ for any fixed set of proposition symbols that includes $A, B, C$.
\end{enumerate}

\paragraph{Sample Answer:}
\begin{verbatim}
a,b,c,d
\end{verbatim}
\end{questions}


%
% Question 3
%
\begin{questions}
\section{Propositional Logic 3}
\question

We denote L0 the set of propositional logic sentences built from a set $\mathcal X$ of $n$ propositional symbols.
we consider the following new formal languages, where some logical connectives are not allowed:
\begin{itemize}
	\item L1 is defined as follows:\\
	True and False are sentences of L1.
	All symbols of $\mathcal X$ are sentences of L1.
	If $s, s'$ are two sentences of L1, then $\neg s$, $(s \wedge s')$, $(s \vee s')$, and $(s \Rightarrow s')$ are four sentences of L1.
	\item L2 is defined as follows:\\
	True and False are sentences of L2.
	All symbols of $\mathcal X$ are sentences of L2.
	If $s, s'$ are two sentences of L2, then $\neg s$, $(s \wedge s')$, and $(s \vee s')$ are three sentences of L2.
	\item L3 is defined as follows:\\
	True and False are sentences of L3.
	All symbols of $\mathcal X$ are sentences of L3.
	If $s, s'$ are two sentences of L3, then $\neg s$ and $(s \wedge s')$ are two sentences of L3.
	\item L4 is defined as follows:\\
	True and False are sentences of L4.
	All symbols of $\mathcal X$ are sentences of L4.
	If $s$ are two sentences of L4, then $\neg s$ is a sentence of L4.
	%\item L5 is defined as follows:\\
	%True and False are sentences of L5.
	%All symbols of $\mathcal X$ are sentences of L5.
	%If $s, s'$ are two sentences of L5 then $s$ and $s \oplus s'$ are two sentences of L4, where $oplus$ represents the exclusive or connective.
\end{itemize}
We consider the following binary relation between languages: L $\subseteq$ L' iff any sentences that can be expressed in L can equivalently be expressed in L'.

Answer "yes" or "no" the following questions.

\begin{enumerate}
\item L1 $\subseteq$ L0
\item L2 $\subseteq$ L0
\item L3 $\subseteq$ L0
\item L4 $\subseteq$ L0
\item L0 $\subseteq$ L1
\item L0 $\subseteq$ L2
\item L0 $\subseteq$ L3
\item L0 $\subseteq$ L4
\end{enumerate}

\paragraph{Sample Answer:}
\begin{verbatim}
no,no,no,no,no,no,no,no
\end{verbatim}
\end{questions}


%
% Question 4
%
\begin{questions}
\section{First-Order Logic 1}
\question

Are the following are valid (necessarily true) sentences?

\begin{enumerate}[a.]
\item $(\exists x\ x=x) \Rightarrow (\forall y \exists z\ y =z)$.
\item $\forall x\ P(x) \vee \neg P(x)$. 
\item $\forall x\ Smart(x)\vee (x=x)$.
\end{enumerate}

Answer "Valid" or "Invalid" the following questions.

\paragraph{Sample Answer:}
\begin{verbatim}
Valid,Valid,Valid
\end{verbatim}
\end{questions}



%
% Question 5
%
\begin{questions}
\section{First-Order Logic 2}
\question

This exercise uses the function $Map\ Color$ and predicates $In(T, y)$, $Borders (x ,y)$, and $Country(x)$, whose arguments are geographical regions, along with constant symbols for various regions. In each of the following we give an English sentence and a number of candidate logical expressions. 

\begin{enumerate}[a.]
	\item Paris and Marseilles are both in France.
	\begin{enumerate}[(i)]
		\item $In(Paris \wedge Marseilles,France)$.
		\item $In(Paris,France) \wedge In(Marseilles,France)$.
		\item $In(Paris,France) \vee In(Marseilles,France)$.
	\end{enumerate}
	\item There is a country that borders both Iraq and Pakistan.
	\begin{enumerate}[(i)]
		\item $\exists c\ Country(c) \wedge Border (c, Iraq) \wedge Border (c,Pakistan)$.
		\item $\exists c\ Country(c) \Rightarrow [Border(c, Iraq) \wedge Border (c,Pakistan)]$.
		\item $[\exists c\ Country(c)] \Rightarrow [Border(c, Iraq) \wedge Border(c,Pakistan)]$.
		\item $\exists c\ Border(Country(c), Iraq \wedge Pakistan)$.
	\end{enumerate}
	\item All countries that border Ecuador are in South America.
	\begin{enumerate}[(i)]
		\item $\forall c\ Country(c) \wedge Border (c,Ecuador ) \Rightarrow In(c, SouthAmerica)$.
		\item $\forall c\ Country(c) \Rightarrow [Border (c,Ecuador ) \Rightarrow In(c, SouthAmerica)]$.
		\item  $\forall c\ [Country(c) \Rightarrow Border (c,Ecuador )] \Rightarrow In(c, SouthAmerica)$.
		\item $\forall c\ Country(c) \wedge Border (c,Ecuador ) \wedge In(c, SouthAmerica)$.
	\end{enumerate}
	\item No region in South America borders any region in Europe.
	\begin{enumerate}[(i)]
		\item $\neg[\exists c, d\ In(c, SouthAmerica) \wedge In(d,Europe) \wedge Borders(c, d)]$.
		\item $\forall c, d\ [In(c, SouthAmerica) \wedge In(d,Europe)] \Rightarrow \neg Borders(c, d)]$.
		\item $\neg \forall c\ In(c, SouthAmerica) \Rightarrow \exists d\ In(d,Europe) \wedge \neg Borders(c, d)$.
		\item $\forall c\ In(c, SouthAmerica) \Rightarrow \forall d\ In(d,Europe) \Rightarrow \neg Borders(c, d)$.
	\end{enumerate}
	\item No two adjacent countries have the same map color.
	\begin{enumerate}[(i)]
		\item $\forall x, y\ \neg Country(x) \vee \neg Country(y) \vee \neg Borders(x, y) \vee \neg (MapColor (x) = MapColor (y))$.
		\item $\forall x, y\ (Country(x) \wedge Country(y) \wedge Borders (x, y) \wedge \neg(x = y)) \Rightarrow \neg(MapColor (x) = MapColor (y))$.
		\item $\forall x, y\ Country(x) \wedge Country(y) \wedge Borders(x, y) \wedge \neg(MapColor (x) = MapColor (y))$.
		\item $\forall x, y\ (Country(x) \wedge Country(y) \wedge Borders (x, y)) \Rightarrow MapColor (x \neq y)$.
	\end{enumerate}
\end{enumerate}

For each of the logical expressions, state whether it...
\begin{enumerate}[1]
\item correctly expresses the English sentence;
\item is syntactically invalid and therefore meaningless; \item is syntactically valid but does not express the meaning of the English sentence.
\end{enumerate}

\paragraph{Sample Answer:}
\begin{verbatim}
233
2333
1222
1333
3222
\end{verbatim}
\end{questions}




\end{document}
