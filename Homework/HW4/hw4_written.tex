\documentclass[11pt, answers]{exam}
\usepackage[margin=1in]{geometry}
\usepackage{amsfonts, amsmath, amssymb, amsthm}
\usepackage{mathtools}
\usepackage{enumerate}
\usepackage{listings}
\usepackage{adjustbox}
\usepackage{cancel}
\usepackage{hyperref}
\usepackage{booktabs}
\usepackage[boxed]{algorithm}
\usepackage[noend]{algpseudocode}
\usepackage{tikz}
\usepackage{float}
\usepackage{setspace}

%
% Basic Document Settings
%

% \topmargin=-0.5in
% \evensidemargin=0in
% \oddsidemargin=0in
% \textwidth=6.5in
% \textheight=9.0in
% \headsep=0.25in

% \linespread{1.1}

\geometry{left=2cm,right=2cm,top=2cm,bottom=2cm}

\pagestyle{headandfoot}
\lhead{\hmwkClass\ : \hmwkType\ \#\hmwkNumber\ (Due \hmwkDue)}
\cfoot{\thepage}
% \renewcommand\headrulewidth{0.4pt}
% \renewcommand\footrulewidth{0.4pt}

\setlength\parindent{0pt}

%
% Create Problem Sections
%
\qformat{\hfill}

\newcommand{\hmwkType}{Written}
\newcommand{\hmwkNumber}{4}
\newcommand{\hmwkClass}{VE 492}
\newcommand{\hmwkDue}{June 17th, 2020 at 11:59pm}


%
% Title Page
%

\title{Homework \hmwkNumber\ \hmwkType}
\date{\hmwkDue}

%
% Various Helper Commands
%

% space of real numbers \R
\newcommand{\R}{\mathbb{R}}

% expected value \EX
\DeclareMathOperator{\EX}{\mathbb{E}}

% For partial derivatives \pderiv{}{}
\newcommand{\pderiv}[2]{\frac{\partial}{\partial #1} (#2)}

% argmax \argmax
\DeclareMathOperator*{\argmax}{arg\,max}

% sign \sign
\DeclareMathOperator{\sign}{sign}

% norm \norm{}
\DeclarePairedDelimiter{\norm}{\lVert}{\rVert}

% Keys
\newcommand{\key}[1]{\fbox{{\sc #1}}}
\newcommand{\ctrl}{\key{ctrl}--}
\newcommand{\shift}{\key{shift}--}
\newcommand{\run}{\key{run} \ }
\newcommand{\runkey}[1]{\run \key{#1}}
\newcommand{\extend}{\key{extend} \ }
\newcommand{\kkey}[1]{\key{k$_{#1}$}}

\begin{document}
\maketitle

\section{Reinforcement Learning}

Imagine an unknown game which has only two states \{A, B\} and in each state the agent has two actions to choose from: \{Up, Down\}. Suppose a game agent chooses actions according to some policy $\pi$ and generates the following sequence of actions and rewards in the unknown game:

\begin{table}[h!]
\centering
\begin{tabular}{|c|c|c|c|c|}
\hline
$t$ & $s_t$ & $a_t$   & $s_{t+1}$ & $r_t$ \\ \hline
0 & A  & Down & B    & -2 \\ \hline
1 & B  & Down & B    & -4 \\ \hline
2 & B  & Up   & B    & 0  \\ \hline
3 & B  & Up   & A    & 3  \\ \hline
4 & A  & Up   & A    & 1  \\ \hline
\end{tabular}
\end{table}

Unless specified otherwise, assume a discount factor $\gamma$ = 0.5 and a learning rate $\alpha$ = 0.5.\\
\begin{enumerate}

\item Recall the update function of Q-learning is:

$$Q\left(s_{t}, a_{t}\right) \leftarrow(1-\alpha) Q\left(s_{t}, a_{t}\right)+\alpha\left(r_{t}+\gamma \max _{a^{\prime}} Q\left(s_{t+1}, a^{\prime}\right)\right)$$

Assume that all Q-values initialized as 0. What are the following Q-values learned by running Q-learning with the above experience sequence?

$$Q(A,Down)=\_\_\_\_, \quad Q(B,Up)=\_\_\_\_$$

\item In model-based reinforcement learning, we first estimate the transition function $T(s,a,s')$ and the reward function $R(s,a,s')$. Fill in the following estimates of T and R, estimated from the experience above. Write "n/a" if not applicable or undefined.

$$\hat{T}(A,Up,A)=\_\_\_\_ \quad \hat{T}(A,Up,B)=\_\_\_\_ \quad \hat{T}(B,Up,A)=\_\_\_\_ \quad \hat{T}(B,Up,B)=\_\_\_\_$$

$$\hat{R}(A,Up,A)=\_\_\_\_ \quad \hat{R}(A,Up,B)=\_\_\_\_ \quad \hat{R}(B,Up,A)=\_\_\_\_ \quad \hat{R}(B,Up,B)=\_\_\_\_$$

\newpage

\item To decouple this question from the previous one, assume we had a different experience and ended up with the following estimates of the transition and reward functions:


\begin{table}[h!]
\centering
\begin{tabular}{|c|c|c|c|c|}
\hline
s & a    & s' & $\hat{T}$(s,a,s') & $\hat{R}$(s,a,s') \\ \hline
A & Up   & A  & 1         & 10        \\ \hline
A & Down & A  & 0.5       & 2         \\ \hline
A & Down & B  & 0.5       & 2         \\ \hline
B & Up   & A  & 1         & -5        \\ \hline
B & Down & B  & 1         & 8         \\ \hline
\end{tabular}
\end{table}

\begin{enumerate}
\item Give the optimal policy $\hat{\pi}^*(s)$ and $\hat{V}^*s$ for the MDP with the transition function $\hat{T}$ and the reward function $\hat{R}$.

Hint: for any $x \in \mathbb{R},|x|<1$, we have $1+x+x^{2}+x^{3}+x^{4}+\cdots=1 /(1-x)$

$$\hat{\pi}^*(A)=\_\_\_\_ \quad \hat{\pi}^*(B)=\_\_\_\_ \quad \hat{V}^*(A)=\_\_\_\_ \quad \hat{V}^*(B)=\_\_\_\_$$

\item If we repeatedly feed this new experience sequence through our Q-learning algorithm, what values will it converge to? Assume the learning rate $\alpha_t$ is properly chosen so that convergence is guaranteed.
\begin{enumerate}
\item the value found above,$\hat{V}^*$
\item the optimal values,$V^*$
\item neither $\hat{V}^*$ nor $V^*$
\item not enough information to determine
\end{enumerate}
\end{enumerate}
\end{enumerate}

\newpage
\section{Policy Evaluation}

In this question, you will be working in an MDP with states S, actions A, discount factor $\gamma$, transition function T, and reward function R.\\

We have some fixed policy $\pi: S \rightarrow A$, which returns an action $a=\pi(s)$ for each state $s \in S$. We want to learn the Q function $Q^{\pi}(s, a)$ for this policy: the expected discounted reward from taking action a in state s and then
continuing to act according to $\pi$:$Q^{\pi}(s, a)=\sum_{s^{\prime}} T\left(s, a, s^{\prime}\right)\left[R\left(s, a, s^{\prime}\right)+\gamma Q^{\pi}\left(s^{\prime}, \pi\left(s^{\prime}\right)\right]\right.$ The policy $\pi$ will not change
while running any of the algorithms below.
\begin{enumerate}
\item Can we guarantee anything about how the values $Q^{\pi}$ compare to the values $Q^{*}$ for an optimal policy $\pi^{*}$?
\begin{enumerate}
\item $Q^{\pi}(s, a) \leq Q^{*}(s, a)$ for all $s,a$
\item $Q^{\pi}(s, a) = Q^{*}(s, a)$ for all $s,a$
\item $Q^{\pi}(s, a) \geq Q^{*}(s, a)$ for all $s,a$
\item None of the above guaranteed
\end{enumerate}

\item Suppose T and R are unknown. You will develop sample-based methods to estimate $Q^{\pi}$. You obtain a series of samples $\left(s_{1}, a_{1}, r_{1}\right),\left(s_{2}, a_{2}, r_{2}\right), \ldots\left(s_{T}, a_{T}, r_{T}\right)$ from acting according to this policy (where $a_t = \pi(s_t)$, for all $t$)
\begin{enumerate}
\item Recall the update equation for the Temporal Difference algorithm, performed on each sample in sequence:
$$V\left(s_{t}\right) \leftarrow(1-\alpha) V\left(s_{t}\right)+\alpha\left(r_{t}+\gamma V\left(s_{t+1}\right)\right)$$

which approximates the expected discounted reward $V^{\pi}(s)$ for following policy $\pi$ from each state s, for a learning rate $\alpha$.
Fill in the blank below to create a similar update equation which will approximate $Q^{\pi}$  using the samples.
You can use any of the terms $Q, s_{t}, s_{t+1}, a_{t}, a_{t+1}, r_{t}, r_{t+1}, \gamma, \alpha, \pi$ in you equation, as well as $\sum$ and max
with any index variables (i.e. you could write $\max _{a}$ or $\sum_{a}$ and then use a somewhere else), but no other terms.

$$Q\left(s_{t}, a_{t}\right) \leftarrow(1-\alpha) Q\left(s_{t}, a_{t}\right)+\alpha[\_\_\_\_\_\_\_\_\_]$$

\item Now, we will approximate $Q^{\pi}$ using a linear function:$Q(s, a)=\sum_{i=1}^{d} w_{i} f_{i}(s, a)$ for weights $w_{1}, \dots, w_{d}$ and feature functions $f_{1}(s, a), \ldots, f_{d}(s, a)$.

To decouple this part from the previous part, use $Q_{samp}$ for the value in the blank in part (a)(i.e. $\left.Q\left(s_{t}, a_{t}\right) \leftarrow(1-\alpha) Q\left(s_{t}, a_{t}\right)+\alpha Q_{s a m p}\right)$
Which of the following is the correct sample-based update for each $w_i$?
\begin{enumerate}
\item $w_{i} \leftarrow w_{i}+\alpha\left[Q\left(s_{t}, a_{t}\right)-Q_{s a m p}\right]$

\item $w_{i} \leftarrow w_{i}-\alpha\left[Q\left(s_{t}, a_{t}\right)-Q_{s a m p}\right]$

\item $w_{i} \leftarrow w_{i}+\alpha\left[Q\left(s_{t}, a_{t}\right)-Q_{s a m p}\right] f_{i}\left(s_{t}, a_{t}\right)$

\item $w_{i} \leftarrow w_{i}-\alpha\left[Q\left(s_{t}, a_{t}\right)-Q_{s a m p}\right] f_{i}\left(s_{t}, a_{t}\right)$

\item $w_{i} \leftarrow w_{i}+\alpha\left[Q\left(s_{t}, a_{t}\right)-Q_{s a m p}\right]w_i$

\item $w_{i} \leftarrow w_{i}+\alpha\left[Q\left(s_{t}, a_{t}\right)-Q_{s a m p}\right]w_i$
\end{enumerate}

\item The algorithms in the previous parts (part a and b) are:
\begin{enumerate}
\item model-based
\item model-free
\end{enumerate}
\end{enumerate}
\end{enumerate}
\end{document}
